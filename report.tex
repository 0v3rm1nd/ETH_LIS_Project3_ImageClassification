\documentclass[a4paper, 11pt]{article}
\usepackage{graphicx}
\usepackage{amsmath}
\usepackage[pdftex]{hyperref}

% Lengths and indenting
\setlength{\textwidth}{16.5cm}
\setlength{\marginparwidth}{1.5cm}
\setlength{\parindent}{0cm}
\setlength{\parskip}{0.15cm}
\setlength{\textheight}{22cm}
\setlength{\oddsidemargin}{0cm}
\setlength{\evensidemargin}{\oddsidemargin}
\setlength{\topmargin}{0cm}
\setlength{\headheight}{0cm}
\setlength{\headsep}{0cm}

\renewcommand{\familydefault}{\sfdefault}

\title{Introduction to Learning and Intelligent Systems - Spring 2015}
\author{Martin Ivanov (ivanovma@student.ethz.ch)\\ Can Tuerk (can.tuerk@juniors.ethz.ch)\\ Jens Hauser(jhauser@student.ethz.ch)\\}
\date{\today}

\begin{document}
\maketitle

\section*{Project 3 : Image Classification}

\subsection*{Problem description}
We received datafiles of images which we had to classify into ten different categories. The data consists of 2048 features and the amount of training data points is 40.000.

\subsection*{Solution}
Our solution with a best peformance of about 0.19 on the validation data due to the given loss function is mainly based on the following approach:

\begin{itemize}

\item To reduce the dimensions of our training data we first used a Principal Component approach with a randomized Singular Value Decomposition which keeps only the most significant singular vectors for the projection.

\item After that we piped the reduced data to an gaussian Support Vector Classifier.
\end{itemize}

The best performance result we got with this approach is based on 500 $Principal Components$ and a penalty parameter $C$ of 10 of the SVM.


\newpage

\subsection*{Different Approaches}
Unfortunately we didn't reach the hard baseline this time, although we spent a huge amount of time and tried the following approaches with a lot of different hyper parameters:

\begin{itemize}
\item ExtraTreesClassifier
\item PCA - ExtraTreesClassifier
\item RandomizedPCA - ExtraTreesClassifier
\item kernelPCA - ExtraTreesClassifier
\item PCA - LinearSVC
\item kernelPCA - LinearSVC
\item SGDClassifier
\item RandomizedPCA - LinearSVC
\item PCA - SGDClassifier
\item RandomizedPCA - SGDClassifier
\item PCA - SVC
\item RandomizedPCA - SVC
\item kernelPCA - SVC
\item Neural Network - LinearSVC
\item Neural Network - SVC

\end{itemize}


\end{document}
